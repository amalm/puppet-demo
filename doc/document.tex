%%This is a very basic article template.
%%There is just one section and two subsections.
\documentclass{article}

\begin{document}

\section{Continious Delivery mit Puppet alt. Konfigurationsmangement mit Puppet}

Eine Vielzahl an Applikationen, jede enwickelt mit Agile Methoden von mehreren Teams.

Gebaut wird stündlich oder öfters mit Continous Integration - in unserem Fall Jenkins\cite{Jenkins}.

Die Applikationen sollen automatisch mit verschiedenen Konfigurationen installiert werden und danach - ebenso automatisch - getestet werden.

Da die Applikationen in unterschiedliche Länder eingesetz werden und für jedes Locale getestet werden sollte, multiplizieren sich die Servers.

Nach hoffentlich erfolgreiche Integrationstests sollten die Applikationen auf weitere Servers in QA- und Produktionsstufe installiert und konfiguriert.

Zur guter Letz werden die Applikationen im Clusters eingesetzt.

Für die Installation und Konfiguration wird je Applikation Scripts entwickelt. Fallweise mit Copy/Paste verfahren und divergierende Weiterentwicklung.

Dass war die Ausgangssituation als wir angefangen haben uns mit dem Thema Konfigurationsmanagement auseinander zu setzen. Wir haben uns für 
Puppet\cite{Puppet} entschieden, eine andere Möglichkeit wäre \cite{Chef}.

\subsection{Ziele}

More plain text.

\subsection{Entwicklungsumgebung}

Plain text.

\subsection{Demo: Apache und eine Webseite mit Puppet installieren}

\author{Michael Haslgrübler}

\author{Anders Malmborg}

\subsection{Referenzen}

\begin{thebibliography}{9}

\bibitem{Jenkins}
  Jenkins: An extendable open source continuous integration server.
  http://jenkins-ci.org/
\bibitem{Puppet}
  Puppet: Puppet is IT automation software that helps system administrators manage infrastructure throughout its lifecycle, 
  from provisioning and configuration to patch management and compliance.
  http://puppetlabs.com/puppet/what-is-puppet/
\bibitem{Chef}
  Chef: Chef is a systems integration framework, built to bring the benefits of configuration management to your entire infrastructure.
  http://wiki.opscode.com/display/chef/Home
  
\end{thebibliography}

\end{document}
